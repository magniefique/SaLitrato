\part{Introduction and its Background}

\hspace\parindent
In today’s digital age, one of the most useful functions of using an app on a computer is its
capability of allowing an individual to find words or phrases quickly in a full text, also known as
the full-text search feature within documents. One popular function that uses the concept of
full-text search is the keyboard shortcut “ctrl+f”. People often use this to find a specific phrase
or characters in a long document or programs such as internet browsers, word processors,
PowerPoint, etc.

\hfill

\textbf{Full-text search} is a method to search and locate a text pattern through the actual content
of your computer documents \texttt{(Full Text Search | FileCenter DMS, 2023; Stasilovich, 2018)}.
However, this feature is accompanied by a significant limitation. In this technique, the documents
must have actual text and not just scanned images. Therefore, if an individual cannot select and
copy a text from the document, the text cannot be searched. This also means that a scanned document,
scanned image, or any type of photo is not searchable even if it has text in it.  

\hfill

Since images also contain valuable information and play a growing significant role in terms of
communication, information sharing, and data representation, the inability to extract text from
images in a full-text search feature makes it more difficult to access valuable information
efficiently. To bridge the gap between text and images, Optical Character Recognition (OCR)
technology was developed by converting all possible text from an image to become machine-readable.
According to Kudan (2023), text recognition technology may also involve translating, processing text
discovered in images, and turning printed documents into editable formats. It may even read text
aloud to those who are blind. Jobs in industries including business, education, healthcare, and
government may be done much more efficiently by using text recognition to automate manual
activities, boost productivity, and provide access to previously inaccessible information. The trend
of storing information from printed materials in digital format has greatly increased in recent
years. This makes it easier to store information and access it when needed, as well as helping to
preserve the information. The field of text recognition is also driven by the need for a more
convenient and quick way of preserving and accessing documents that contain information. One of the
easiest ways to transfer information from paper or books is to scan them. This turns the information
into an image and prevents the text from being reproduced (Adyanthaya, 2020).

\hfill

The feature that uses OCR technology to extract text from images and lets users search specific
words, characters, or phrases within the extracted text from images is called the \textbf{in-image
text searching feature}. The main distinction between the full-text search and in-image text search
lies in the source of the text being searched. In-image text search searches for text specifically
from images while full-text search only recognizes text from text-based content such as documents or
articles. There are numerous valuable applications of \textbf{in-image text searching} across
various industries. One instance is it enables an efficient method for converting physical documents
into digital and searchable formats. An individual only needs to scan physical documents such as
books, contracts, and receipts and upload them to software capable of in-image text searching. This
method allows faster retrieval of information and reduces the need for manually typing the content
of the physical documents on a computer. The feature can also be useful in content moderation.
Social media applications and platforms can use this feature to scan images for text content, then
process the extracted text to detect any harmful or inappropriate text that violates their
platform’s policies. Furthermore, in-image text search is also notable for educational and learning
purposes; students, teachers, and researchers can use the feature to take notes and search specific
terms efficiently.

\hfill

Various software with OCR and the capability to extract text within images already exist. However,
there are still fewer studies online about applications that provide users with an added feature of
searchable text within the extracted content from the images. The full-text search feature is still
the most common one to be used in applications. Based on the resources and studies the researchers
have gathered online, the text-searching feature is more implemented in text-based content over
images. Furthermore, studies online often used the Boyer-Moore algorithm to implement their search
function. For instance, Waruwu (2017), Buslim et al. (2020), Khumaidi et al. (2020), and Layustira
and Istiono (2021) did use the Boyer-Moore algorithm but did not implement the algorithm for an
in-image text search feature. These studies did not allow users to search for text patterns within
the content of images for their application software. Additionally, in the study of Fitriyah et al.
(2020), the Boyer-Moore algorithm was implemented in a web-based computer and informatic terms
dictionary. In the study of Arini and Suseno (2019), Boyer-Moore Algorithm was applied to a
dictionary of web-based information technology terms. Furthermore, the algorithm was also used in
the application of dictionary livestock terms (Saputri et al., 2018). Based on the mentioned studies
that use the Boyer-Moore algorithm for text-search features, the algorithm, and the feature is often
used in dictionaries and other text-based content or forms.

\hfill

As evident in the studies and resources online, Boyer-Moore Algorithm is commonly used for search
functions in text-based content, such as dictionaries and documents. Thus, there is a lack of
studies and resources online about applications that offer in-image text searching while also
utilizing the Boyer-Moore algorithm as its string-matching algorithm. To address the aforementioned
issue, the study will focus on developing an application software capable of \textbf{folder-wide detecting
and locating a text pattern within multiple image files}, enabling users to efficiently locate
information present in images.

\hfill

In terms of implementing a text search feature into application software, it is typical to utilize a
string-matching algorithm, such as the Boyer-Moore Algorithm, KMP Algorithm, Naïve Algorithm, and
Brute Force Algorithm. \textbf{String-matching algorithms} are algorithms that find occurrences or matches of
a pattern within a text or content of multiple files (GeeksforGeeks, 2022). Since numerous
string-matching algorithms can be used, selecting one depends on the nature and requirements of the
software. As such, the researchers will utilize the \textbf{Boyer-Moore algorithm} to implement the in-image
text search feature on the software.

\hfill

The key feature of the Boyer-Moore Algorithm is its efficiency in pattern searching within huge
bodies of text. As the number of images uploaded in the in-image text search function increases, the
larger the block of text will be extracted. Thus, the number of texts to be traversed by the
algorithm increases too. Boyer-Moore Algorithm is well-suited for this scenario as it works best
with huge bodies of text. Furthermore, according to several studies online, the \textbf{Boyer-Moore
Algorithm} is the more efficient algorithm compared to several other string-matching algorithms.
(Khumaidi et al., 2020; Layustira \& Istiono, 2021; Rasool et al., 2012; Supatmi \& Sumitra, 2019;
Dawood \& Barakat, 2020; Lin \& Soe, 2020; Buslim et al., 2020; Borah et al., 2013). Additionally,
numerous studies online that implemented search functions using the Boyer-Moore algorithm have also
proven the efficiency of the algorithm (Fitriyah et al., 2020; Arini \& Suseno, 2019; Saputri et
al., 2018).

\hfill

Since the algorithm for the study has been decided, other features will be implemented into the
software. The application will be integrated with an optical character recognition (OCR) API to
recognize and extract texts from images. The software will also implement a function that highlights
the location of the text pattern in the extracted text or shows a message if the text pattern is not
detected.
